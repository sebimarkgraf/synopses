\section{Maximum Flows \& Matchings}%
\label{mfm:sec:maximum_flows_matchings}

\begin{itemize}
	\item \textbf{Networks}:
	\begin{itemize}
		\item Gerichtete, gewichtete Graphen mit einer \textbf{Source Node s} (Quelle) und einer \textbf{Sink Node t} (Senke)
		\item In die \textbf{Source Node} gehen keine Kanten \textbf{ein}, von der \textbf{Sink Node} gehen keine Kanten \textbf{aus}
		\item Kantengewichte bezeichnet man auch als \textbf{Kapazität} einer Kante
	\end{itemize}
	\item \textbf{Flows}:
	\begin{itemize}
		\item Funktion $f$, die jeder Kante einen Wert \textbf{zwischen 0 und der Kapazität} zuordnet
		\item Wert eines Flows \textbf{val(f)} entspricht dem gesamten ausgehenden Flow der aus $s$ kommt und in $t$ mündet
		\item \textbf{Metapher}: Netzwerk als Rohrsystem zum Wassertransport
		\item \textbf{Blocking Flows} erzeugen auf jedem möglichen Pfad von s zu t eine komplett gesättigte Kante
		\item \textbf{Flow Conservation} bedeutet, dass \textbf{nur die Quelle Flow erzeugt} und \textbf{nur die Senke Flow aufnimmt}
		\item \textbf{Ziel}: Flow mit maximalem Wert finden (Transportmaximierung)
	\end{itemize}
	\item \textbf{s-t Cuts}
	\begin{itemize}
		\item Partitionierung der Knoten in Teilmengen $S$ und $T$ mit $s \in S, t \in T$
		\item Die \textbf{Kapazität des Cuts} entspricht der Summe der Kantengewichte aller Kanten die in einem S-Knoten starten und in einem T-Knoten münden
		\item Die minimale Kapazität eines s-t Cuts entspricht dem Wert eines s-t \textbf{Max Flows}
	\end{itemize}
\end{itemize}

\subsection{Maximierung von Flow-Werten}%
\label{mfm:sub:maximierung_von_flow_werten}

\begin{itemize}
	\item \textbf{Linear Programming}: Möglich, aber geht besser
	\item \textbf{Ford Fulkerson Algorithmus (Augmenting Paths)}:
	\begin{itemize}
		\item \textbf{Idee}: Finde einen Pfad von Quelle zu Senke, entlang dessen der Fluss vergrößert werden kann, ohne die Kapazitäten zu überschreiten (augmentierender Pfad)
		\item \textbf{Vorgehen}:
		\begin{enumerate}
			\item Finde (per BFS) einen möglichen Pfad von s zu t (keine Kante reizt ihre Kapazität komplett aus)
			\item Sende einen Flow durch den Pfad und füge diesen dem Gesamtflow hinzu
			\item Passe die Kapazitäten aller Kanten an (erzeugt \textbf{Residualgraph}) und wiederhole den Prozess mit dem neuen Graph
			\item Sobald es keinen augmentierenden Pfad mehr gibt, ist der Gesamtflow der \textbf{Max Flow}
		\end{enumerate}
		\item \textbf{Anmerkung}: Beim Finden eines möglichen Pfades darf auch Flow \textbf{entgegen einer Kante fließen}, wodurch der Flow auf dieser Kante \textbf{reduziert} wird
	\end{itemize}
	\newpage
	\item \textbf{Dinitz Algorithmus}
	\begin{itemize}
		\item \textbf{Idee}: Sende so lange Flows, bis ein \textbf{Blocking Flow} erreicht wurde
		\item \textbf{Vorgehen}:
		\begin{enumerate}
			\item Generiere (per BFS) einen \textbf{Level Graph} (Graph, der jedem Knoten einen Level zuteilt, welcher der kürzesten Distanz zur Quelle entspricht) und teste, ob mehr Flow möglich ist
			\item Falls mehr Flow möglich ist, finde einen \textbf{Blocking Flow} im \textbf{Level Graph} und füge diesen dem Gesamtflow hinzu, dann wiederhole 1.; ansonsten \texttt{return}
		\end{enumerate}
	\end{itemize}
\end{itemize}

\subsection{Berechnung von Blocking Flows}%
\label{mfm:sub:berechnung_von_blocking_flows}

\begin{itemize}
	\item \textbf{Idee}: Wiederholte Tiefensuche
	\item \textbf{Vorgehen}:
	\begin{enumerate}
		\item Starte eine Tiefensuche bei der Quelle
		\item Ist der nächste Knoten die Senke, so berechne den Flow und falls irgendeine Kante komplett ausgereizt wird
		\begin{enumerate}
			\item füge den \textbf{Blocking Flow} zur Liste hinzu
			\item \textbf{entferne} die Kante
			\item starte den Prozess \textbf{von vorne}
		\end{enumerate}
		\item Ist eine unbesuchte Kante verfügbar, gehe weiter (\texttt{extend})
		\item Ist keine unbesuchte Kante verfügbar, gehe eine Kante zurück (\texttt{retreat})
		\item Sobald ein \texttt{retreat} zum Startknoten stattgefunden hat, ist die Suche beendet
	\end{enumerate}
\end{itemize}

\subsection{Matching}%
\label{mfm:sub:matching}

\begin{itemize}
	\item \textbf{Definition}: Eine Teilmenge $M$ der Kantenmenge ist ein \textbf{Matching}, wenn $(V, M)$ maximal Grad 1 hat
	\item Ein Matching ist \textbf{maximal}, wenn das Hinzufügen einer beliebigen weitere Kante aus der restlichen Kantenmenge die Matching-Eigenschaft verletzen würde
	\item Ein Matching hat \textbf{maximale Kardinalität}, wenn es kein Matching mit einer größeren Kardinalität gibt
\end{itemize}

\newpage
\subsection{Preflow-Push Algorithmen}%
\label{mfm:sub:preflow_push_algorithmen}

\begin{itemize}
	\item \textbf{Definition}: Ein Preflow ist ein Flow mit gelockerter \textbf{Flow Conservation} - Knoten dürfen \textbf{Teile des Flows verbrauchen} (\textbf{Exzess)} und gelten, falls sie dies tun, als \textbf{aktiv}
	\item \textbf{Ziel}: Finden eines \textbf{Max Flows}
	\item \textbf{Push-Funktion}: \quotestyle{Verschieben des Exzesses in Richtung von t}
	\begin{itemize}
		\item Nimmt eine Kante $(v,w)$ und einen Wert $\delta$, der höchstens dem Exzess von $v$ entspricht
		\item Zieht $\delta$ vom Exzess von $v$ ab und addiert ihn auf den Exzess von $w$
		\item Falls die Kante \textbf{entgegen dem Flow} läuft, wird $\delta$ von diesem abgezogen, ansonsten \textbf{auf diesen aufaddiert}
	\end{itemize}
	\item \textbf{Level-Funktion}:
	\begin{itemize}
		\item $d(v)$ enthält eine Approximation der Distanz von $v$ zu $t$
		\item $d(s) = n$ und $d(t) = 0$ und $d(v) \leq d(w) + 1 \forall$ Kanten $(v,w)$
		\item Die Level-Funktion ordnet Kanten $(v,w)$ eine Richtungseigenschaft zu:
		\begin{itemize}
			\item \texttt{steep}: $d(w) < d(v) - 1$
			\item \texttt{downward}: $d(w) < d(v)$
			\item \texttt{horizontal}: $d(w) = d(v)$
			\item \texttt{upward}: $d(w) > d(v)$
		\end{itemize}
	\end{itemize}
	\item \textbf{Preflow-Flush}:
	\begin{enumerate}
		\item Pushe alle Kanten mit der Kapazität der Kante als $\delta$
		\item Setze $d(v)$ auf $0$ für alle Knoten außer $s$, setze $d(s) = n$
		\item Solange ein Knoten mit Exzess existiert, der nicht $s$ oder $t$ ist
		\begin{itemize}
			\item Falls die Kante \texttt{downward} verläuft: Verschiebe Exzess per \texttt{push} mit $\delta \leq min\{excess(v), c_e^f\}$
			\item Ansonsten: Inkrementiere $d(v)$ (\texttt{relabel})
		\end{itemize}
	\end{enumerate}
\end{itemize}