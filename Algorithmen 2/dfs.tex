\section{Depth-First-Search (DFS, Tiefensuche)}%
\label{dfs:sec:depth_first_search}

\begin{itemize}
	\item \textbf{Idee}: Verfolge einen einzelnen Pfad vollständig, bevor abzweigende Pfade betrachtet werden (statt direkt alle benachbarten Knoten zu beschreiten)
	\item \textbf{Vorgehen}:
	\begin{enumerate}
		\item Initialisiere alle Knoten als nicht markiert
		\item Für alle Startknoten: Markiere den Startknoten und benutze ihn als Wurzel für einen neuen Tiefensuche-Baum; starte die Tiefensuche
		\item \textbf{Tiefensuche}: Verfolge die erstbeste Kante, markiere ggf. den Zielknoten der Kante und beginne die Tiefensuche von dort erneut; sobald es keine Kanten mehr gibt, nutze \textbf{Backtracking} bis es wieder passende Kanten gibt
	\end{enumerate}
	\item \textbf{Übliche Funktionen}:
	\begin{itemize}
		\item \texttt{init} zur Initialisierung
		\item \texttt{root(s)} zur Erstellung eines neuen Tiefensuche-Baumes
		\item \texttt{traverseNonTreeEdge(v,w)} beim Traversieren zu einem bereits markierten Knoten, ansonsten \texttt{traverseTreeEdge(v,w)}
		\item \texttt{backtrack(u,v)} für Backtracking
	\end{itemize}
\end{itemize}

\subsection{Starke Zusammenhangskomponenten}%
\label{dfs:sub:starke_zusammenhangskomponenten}

\begin{itemize}
	\item \textbf{Beobachtung}: Es existiert eine Äquivalenzrelation, deren Äquivalenzklassen alle Knoten sind, die paarweise direkt oder indirekt über Kanten erreichbar sind
	\begin{center}
		\textit{\quotestyle{Kommt man irgendwie von A nach B und von B nach A, so sind A und B in derselben Äquivalenzklasse.}}
	\end{center}
	\item Die Äquivalenzklassen werden als \textbf{starke Zusammenhangskomponenten (strongly connected components, SCCs)} bezeichnet
	\item \textbf{Schrumpfgraph}: Graph, der die SCCs seines Supergraphs als Knoten zusammenfasst. Wir beobachten:
	\begin{itemize}
		\item Schrumpfgraphen sind \textbf{azyklisch}
		\item Hinzufügen einer neuen Kante \textbf{innerhalb eines SCCs}:\\$\Rightarrow$ Schrumpfgraph unverändert
		\item Hinzufügen einer neuen Kante \textbf{zwischen zwei SCCs}, die \textbf{keinen} Kreis erzeugt:\\$\Rightarrow$ \textbf{neue Kante im Schrumpfgraph}
		\item Hinzufügen einer neuen Kante \textbf{zwischen zwei SCCs}, die \textbf{einen} Kreis erzeugt:\\$\Rightarrow$ SCCs auf dem Kreis \textbf{kollabieren}
	\end{itemize}
\end{itemize}

\subsection{Bestimmung starker Zusammenhangskomponenten über DFS}%
\label{dfs:sub:bestimmung_starker_zusammenhangskomponenten_ueber_dfs}

\textbf{Definitionen}:
\begin{itemize}
	\item $V_C$ sind die \textbf{markierten Knoten}, $E_C$ sind die \textbf{bisher explorierten Kanten}
	\item Unmarkierte Knoten gelten als \textbf{nicht erreicht}
	\item \textbf{Offene Komponenten} haben aktive Knoten, andernfalls gelten diese als \textbf{abgeschlossen}
	\item \texttt{component[w]} gibt Repräsentanten einer SCC an
	\item \texttt{oReps} ist ein Stapel von Repräsentanten offener Komponenten
	\item \texttt{oNodes} ist ein Stapel aller offenen Knoten
\end{itemize}
\textbf{Vorgehen bei der Tiefensuche}:
\begin{enumerate}
	\item \textbf{Initialisierung}:\\\texttt{oReps}, \texttt{oNodes} und das \texttt{component}-Array werden geleert
	\item \textbf{Festlegen einer neuen Wurzel - root(s)}:\\Schiebe die neue Wurzel auf die \texttt{oReps} und \texttt{oNodes} Stapel
	\item \textbf{Traversieren zu einem nicht-markierten Knoten - traverseTreeEdge(v,w)}:\\Schiebe den Zielknoten $w$ auf die \texttt{oReps} und \texttt{oNodes} Stapel
	\item \textbf{Traversieren zu einem markierten Knoten - traverseTreeEdge(v,w)}:\\Falls der Zielknoten $w$ nicht in \texttt{oNodes} liegt, passiert nichts. Falls doch, entferne per \texttt{pop} alle Stapelelemente von \texttt{oReps} bis das oberste Element nicht mehr äquivalent zu $w$ ist
	\item \textbf{Backtracking - backtrack(u,v)}:\\Falls der Zielknoten nicht der \texttt{top} von \texttt{oReps} entspricht, passiert nichts. Falls doch, entferne per \texttt{pop} das oberste Element von \texttt{oReps} und führe \texttt{component[w] = v} solange aus, bis $w = v$ gilt.
\end{enumerate}
$\Rightarrow$ \textbf{am Ende sollten beide Stapel leer sein, anhand von component[w] kann man nun die SCCs ablesen}