\section{Prioritätslisten}%
\label{pl:sec:prioritaetslisten}

\begin{itemize}
	\item \textbf{Prioritätsliste (priority queue):}
	\begin{itemize}
		\item Queue, deren Elemente zusätzlich einen Schlüssel besitzen, der die Reihenfolge der Abarbeitung bestimmt
		\item Grundlegende Operationen: \texttt{build, size, insert, min, deleteMin}
	\end{itemize}
	\item \textbf{Adressierbarkeit}:
	\begin{itemize}
		\item Zugriff auf einzelne Elemente per Handle
		\item Wichtig für Algorithmen, bei denen sich die Schlüssel innerhalb der Queue ändern (v.a. Greedy-Algorithmen)
		\item Neue Operationen: \texttt{remove, decreaseKey, merge}
	\end{itemize}
	\item \textbf{Verwendung}: Dijkstra, Jarnik-Prim, Graphpartitionierung, Disk Scheduling uvm.
	\item \textbf{Aufbau}:
	\begin{itemize}
		\item Wald heap-geordneter Bäume (Verallgemeinerung eines Binary Heaps mit einem Wald statt einem Baum und beliebigen Knotengraden)
		\item Wälder lassen sich per \texttt{cut} trennen und per \texttt{link} vereinen
		\item Grundlegender Heap kann variieren, z.B Pairing Heap, Fibonacci Heap (folgt)
	\end{itemize}
	\end{itemize}

	\subsection{Pairing Heaps}%
	\label{pl:sub:pairing_heaps}

	\begin{itemize}
		\item Jedes neue Element ergibt einen neuen Baum, der zum Wald hinzugefügt wird (ggf. wird der minPtr aktualisiert)
		\item \texttt{decreaseKey} aktualisiert den Key des Elements sowie ggf. den minPtr; ist das Element keine Baumwurzel, so wird der Teilbaum per \texttt{cut} herausgeschnitten und als neuer Baum hinzugefügt
		\item \texttt{deleteMin} entfernt das Minimum, aktualisiert den minPtr, macht aus jeden Kindknoten einen neuen Baum und führt paarweise \texttt{union} auf allen Bäumen im Wald aus
		\item \texttt{merge} fügt die Wälder zusammen, aktualisiert den minPtr und leert die zweite Prioritätsliste
		\item \textbf{Nur bei deleteMin wird tatsächlich paarweise union aufgerufen!}
		\item \textbf{Aufbau} als doppelt verkettete Liste
	\end{itemize}

	\newpage
	\subsection{Fibonacci Heaps}%
	\label{pl:sub:fibonacci_heaps}
	
	\begin{itemize}
	\item \textbf{Rang}: Anzahl direkter Kinder (\textbf{Zweck}: \texttt{union} nur für gleichrangige Wurzeln)
	\item \textbf{Kaskadierende Schnitte}: Knoten, die ein Kind verloren haben, werden markiert; markierte Knoten werden geschnitten
	\item \textbf{Aufbau} als doppelt verkettete Liste; zusätzliche Felder in Elementen für Rang und Markierung
	\item \texttt{insert} und \texttt{merge} wie bei Pairing Heaps
	\item \texttt{deleteMin} per \textbf{Union-by-Rank}:
	\begin{enumerate}
		\item Entferne Minimum, mache aus jedem Kind-Teilbaum einen neuen Baum im Wald
		\item Bilde \texttt{union} für alle gleichrangigen Bäume bis nicht mehr möglich (vereinte Bäume mit neuem, höheren Rang sind u.U. erneut vereinbar)
		\item Aktualisiere den minPtr
	\end{enumerate}
	\item \texttt{decreaseKey} mit \textbf{kaskadierenden Schnitten} aktualisiert den Schlüssen und ruft \texttt{cascadingCut} auf dem Element auf
	\item \texttt{cascadingCut}:
	\begin{enumerate}
		\item Entferne die Markierung des Elements, rufe \texttt{cut} auf dem Element auf
		\item Falls der Elternknoten markiert ist, rufe \texttt{cascadingCut} auf dem Elternknoten auf, ansonsten markiere den Elternknoten
		\item \textbf{Zweck}: Hält \texttt{maxRank}-Funktion für \textbf{Union-by-Rank} logarithmisch, verbessert also die Laufzeit
	\end{enumerate}
	\item Fibonacci-Heap, da eine Wurzel $v$ mit $rank(v) = i$ mindestens $F_{i + 2}$ Elemente enthält, wobei $F_i$ die i-te Stelle der Fibonacci-Folge ist
	\end{itemize}

	\subsection{Bucket Queues}%
	\label{pl:sub:bucket_queues}
	
	\begin{itemize}
		\item \textbf{Idee}: Prioritätsliste dem ein zyklisches Array zugrunde liegt; Elemente gleicher Priorität teilen sich dieselbe Arrayzelle
		\item \textbf{Funktionsweise}:
		\begin{itemize}
			\item Zyklisches Array $B$ von $C + 1$ doppelt verketteten Listen (Buckets)
			\item \texttt{insert}: Füge Element mit Priorität $p$ in Zelle $p\ mod (C + 1)$ ein
			\item \texttt{decreaseKey}: Entferne Element aus seiner Liste und füge es mit der geänderten Priorität wieder ein
			\item \texttt{deleteMin}: Beginne bei Bucket $min$; falls dieser leer ist, inkrementiere $min$ und wiederhole; entferne ein beliebiges Element aus dem ersten nicht-leeren Bucket
		\end{itemize}
		\item Nützlich für \textbf{Dijkstras Algorithmus}
	\end{itemize}

	\subsection{Radix Heaps}%
	\label{sub:radix_heaps}
	
	\begin{itemize}
		\item \textbf{Idee}: Bucket Queues, bei denen nicht alle Buckets dieselbe Größe haben
		\item \textbf{Funktionsweise}:
		\begin{itemize}
			\item Verwende Buckets $-1$ bis $K = 1 + \floor{log C}$
			\item Wähle $min$ als das zuletzt aus der Queue entfernte Element
			\item \texttt{insert}: Betrachte binäre Repräsentation des Elements; speichere das Element $v$ in $B(i)$, falls sich $v$ und $min$ zuerst an der i-ten Stelle unterscheiden
			\begin{itemize}
				\item Die gesuchte Stelle ist das \textbf{höchstwertige unterschiedliche} Bit!
				\item Beginne von rechts ab $0$ zu zählen; falls identisch, ist das Ergebnis $-1$
				\item Beispiel: $$\begin{array}{c c c c c c c c}
					1&1&0&0&1&0&0&0\\
					1&1&0&0&0&1&1&1
				\end{array}$$ $\Rightarrow$ Unterschied an Stelle $3$
			\end{itemize}
			\item \texttt{deleteMin}:
			\begin{itemize}
				\item Falls die Liste an Stelle $-1$ leer ist:
				\begin{enumerate}
					\item Finde den kleinsten Index $i$ an dem eine nicht-leere Liste steht
					\item Bewege das Minimum dieser Liste zu $B(-1)$ und zu $min$
					\item Rufe für alle übrigen Elemente aus $B(i)$ \texttt{insert} auf
				\end{enumerate}
				\item Entferne und gebe das erste Element der Liste bei Index $-1$ zurück
			\end{itemize}
		\end{itemize}
	\end{itemize}