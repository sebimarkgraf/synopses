\section{Räumliche Datenstrukturen / Beschleunigungsstrukturen}%
\label{ds:sec:raeumliche_datenstrukturen_beschleunigungsstrukturen}

\subsection{Motivation}%
\label{ds:sub:motivation}

\begin{itemize}
	\item Raytracing ist aufwendiger Prozess, großer Optimierungsbedarf
	\item Großer Teil der Geometrie in einer Szene wird von einem Strahl nicht getroffen
	\item Nötige Intersektionstests sollen auf ein Minimum reduziert werden
	\item \textbf{Idee}: Räumliche Datenstrukturen (Beschleunigungsstrukturen), mit denen Geometrie frühzeitig ignoriert werden kann
	\item \textbf{Anmerkung}: Die Effizient einer bestimmten räumlichen Datenstruktur ist \textbf{immer szenenabhängig!}
\end{itemize}

\subsection{Bounding Volumes}%
\label{ds:sub:bounding_volumes}

\begin{itemize}
	\item \textbf{Idee}: Umhülle komplexe Geometrie mit approximierender, simpler Geometrie; überprüfe umhüllte Geometrie \textbf{erst dann, wenn das Bounding Volume geschnitten wird}
	\item \textbf{Vorteil}: Einfache Intersektionstests (vgl. Mesh aus tausenden Dreiecken vs. Würfel)
	\item \textbf{Nachteil}: Nur grobe Approximation, manchmal false positives
	\item Kugel
	\begin{itemize}
		\item meist unpassend, da zu groß
	\end{itemize}
	\item \textbf{Achsenparallele Box (axis-aligned bounding box, AABB)}
	\begin{itemize}
		\item Sehr einfache Intersektionsberechnung, 6 Ebenengleichungen (Kollision zwischen AABBs noch simpler über Achsen)
		\item Muss bei Rotation des Objektes ggf. skaliert werden
		\item Einfach optimierbar (auch: SIMD)
		\item Standardoption
	\end{itemize}
	\item Orientierte Bounding Box (OBB)
	\begin{itemize}
		\item Aufwendiger als AABBs
		\item Besser approximierend, da nicht achsenorientiert
	\end{itemize}
	\item Slabs
\end{itemize}

\newpage
\subsection{Bounding Volume Hierarchies}%
\label{ds:sub:bounding_volume_hierarchies}

\begin{itemize}
	\item \textbf{Idee}: Baue Hierarchie von Bounding Volumes (umhülle Bounding Volumes mit Bounding Volumes) für weitere Approximation; wird ein Volumen geschnitten, so prüft man daraufhin alle Kindvolumen etc.
	\item \textbf{Vorteil}: Adaptivität, noch schnellere Suche, leicht zu traversieren und konstruieren
	\item \textbf{Nachteil}: Unterteilung muss sinnvoll gewählt sein, Overhead von zusätzlichen Bounding Volumes summiert sich auf
	\item \textbf{Aufbaustrategie}: Umhülle alle Objekte mit einem Bounding Volume und teile dieses dann rekursiv auf
	\item \textbf{Unterteilungskriterien}:
	\begin{itemize}
		\item Mittiges Unterteilen an der Achse mit der größten Ausdehnung
		\item Rotation durch Achsen (unterteile entlang x, dann y, dann z, dann x ...)
		\item Objektzahl in beiden Child-Volumes gleich groß halten
		\item Szenengraph bzw. Modellhierarchie verwenden
		\item Minimieren einer Kostenfunktion (z.B Surface Area Heuristic)
	\end{itemize}
\end{itemize}

\subsection{Reguläre Gitter}%
\label{ds:sub:regulaere_gitter}

\begin{itemize}
	\item \textbf{Idee}: Statt einzelnen Volumen um Geometrie, unterteile kompletten Raum in gleich große Zellen (quasi AABBs); Objekte über mehrere Zellen hinweg werden in jeder dieser Zellen eingetragen
	\item \textbf{Vorteil}: Sehr simpel, einfach zu konstruieren und traversieren, gut geeignet für Szenen mit gleichmäßig verteilter Geometrie
	\item \textbf{Nachteil}: Richtige Gittergröße ggf. schwer zu finden, ungeeignet für komplexe Szenen mit viel Geometrie in wenigen Zellen, nicht adaptiv
	\item \textbf{Mailboxing}: Vermeidung von mehreren Schnitttests mit demselben Objekt, welches sich über mehrere Zellen erstreckt, durch Markieren des Objektes als getestet; Schnittpunkt wird gespeichert
\end{itemize}

\subsection{Octrees}%
\label{ds:sub:octrees}

\begin{itemize}
	\item \textbf{Idee}: Adaptive Gitter; baue reguläres Gitter und unterteile Zellen rekursiv, falls zu viele Primitive enthalten sind (Unterteilung in gleich große Achtelblöcke in 3D, in Viererblöcke in 2D für einen Quadtree)
	\item \textbf{Vorteil}: Adaptiv, feine Unterteilung nur dort wo die Geometrie ist, leicht zu konstruieren und traversieren
	\item \textbf{Nachteil}: Häufige Vertikalbewegung (auf und ab in der Hierarchie), Traversierungskosten zweier Strahlen können stark variieren (Problem bei Parallelität)
\end{itemize}

\subsection{BSP-trees / kd-trees}%
\label{ds:sub:bsp_trees_kd_trees}

\begin{itemize}
	\item \textbf{Idee}: Rekursives binäres Unterteilen des Raumes durch Ebenen (k-dimensionale Binärbäume, kd-trees)
	\item \textbf{Unterschied}: Binary-Space-Partitioning-Trees (BSP-trees) verwenden beliebig orientierte Ebenen, kd-Bäume sind ein Spezialfall mit Ebenen senkrecht zur x-, y- oder z-Achse; kd-trees werden öfter genutzt, da BSPs aufgrund des größeren Suchraums für Teilungsebenen \textbf{schwerer zu konstruieren} sind
	\item \textbf{Aufbaustrategie}: Initialisiere Wurzelknoten um alle Objekte, unterteile dann rekursiv bis Unterteilungskriterium erfüllt
	\begin{itemize}
		\item Primitivenzahl unterschreitet Threshold
		\item Feste maximale Rekursionstiefe ist erreicht
	\end{itemize}
	\item \textbf{Kriterien für die Position der Teilungsebene}:
	\begin{itemize}
		\item Räumliches Mittel (\textbf{spatial median}) entlang der Achse mit der größten Ausdehnung \textbf{oder} rotierend x, dann y, dann z, dann wieder x etc.
		\item Objektmittel (\textbf{object median}): Primitivzahl in beiden Kindknoten soll sich höchstens um 1 unterscheiden
		\item Kostenfunktion (z.B \textbf{Surface Area Heuristic}): Unterteilung nach Oberfläche der Primitiven
	\end{itemize}
	\item \textbf{Vorteil}: Adaptiv, leicht zu traversieren, manchmal performanter als BVHs
	\item \textbf{Nachteil}: Konstruktion aufwendig
\end{itemize}

\subsection{Praktische Anwendung}%
\label{ds:sub:praktische_anwendung}

\begin{itemize}
	\item Oftmals \textbf{Kombination} von verschiedenen Beschleunigungstechniken (z.B kd-trees mit BVHs)
	\item BVHs sehr gut für Szenen die sich \textbf{bewegen} (Alternativen nicht adaptiv genug oder müssen für gute Approximation oft neu generiert werden)
	\item kd-tree wird BSP-tree oft vorgezogen, da Konstruktionskosten den möglichen Performancegewinn selten rechtfertigen
	\item Gitter eher selten, hierarchische Gitter und Octrees eher für Simulationen
	\item \textbf{Anmerkung}: Auch Verwendung \textbf{abseits von Beschleunigung} für Raytracing, z.B Culling, Sortierung, 3D Texturen, Physiksimulation, Robotik
\end{itemize}