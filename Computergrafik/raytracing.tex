\section{Raytracing}%
\label{rt:sec:raytracing}

\subsection{Grundlagen}%
\label{rt:sub:grundlagen}

\begin{itemize}
	\item Bildsynthese durch \textbf{Simulation des Lichttransports}
	\item System ähnlich einer Lochkamera, definiert durch Position, Blickrichtung und vertikaler Orientierung (up-Vektor)
	\item Emittieren von \glqq Lichtstrahlen\grqq\ durch jeden Pixel, zurückverfolgen von der Kamera aus
	\item Prüfen auf Intersektion mit vorhandener Geometrie; sollte eine Intersektion gefunden werden:
		\begin{itemize}
			\item Wähle das getroffene Objekt und dessen Material-Eigenschaften $\Rightarrow$ Shading-Berechnungen
			\item ggf. Verfolgen von weiteren Strahlen ab einer Intersektion (z.B bei spiegelndem Material)
		\end{itemize}
\end{itemize}

\subsection{Ray Generation - Mathematische Aspekte}%
\label{rt:sub:ray_generation_mathematische_aspekte}

\begin{itemize}
	\item Strahl wird durch eine Startposition und eine normierte Richtung modelliert, die um $t \in \realnumbers$ skaliert wird
	\item Allgemein für einen Strahl $r$: $r = e + t * d$ mit der Startposition e und der Richtung d
	\item Bei der Suche nach einem Schnittpunkt wird i.d.R. ein passendes t ausgerechnet, sodass r auf den Schnittpunkt zeigt
\end{itemize}

\subsubsection{Baryzentrische Koordinaten}%
\label{rt:ssub:baryzentrische_koordinaten}

\textbf{Annahme:}
\begin{itemize}
	\item Es existieren k Punkte $P_1, ..., P_k \in \realnumbers^n, k \leq n + 1$
	\item Existiert dann ein Punkt Q der Form $Q = \lambda_1P_1 + \lambda_2P_2 + ... + \lambda_kP_k, \lambda_1 + \lambda_2 + ... + \lambda_k = 1$ (\glqq Affinkombination\grqq)
\end{itemize}
Dann definiert man $(\lambda_1, \lambda_2, ..., \lambda_k)$ als die baryzentrischen Koordinaten von Q bzgl. der Basispunkte $P_1, ..., P_k$. Hiermit lässt sich z.B ein Punkt innerhalb eines Dreiecks anhand der Eckpunkte beschreiben (z.B um zu testen ob ein Punkt in einem Dreieck liegt).\\

\textbf{Berechnung in Dreieck}\\
Berechne $\lambda_1, \lambda_2, \lambda_3$ durch \\
$$\lambda_1 = \frac{A_{\Delta}(Q, P_2, P_3)}{A_\Delta(P_1, P_2, P_3)} \qquad \lambda_2 = \frac{A_{\Delta}(P_1, Q, P_3)}{A_\Delta(P_1, P_2, P_3)} \qquad \lambda_3 = \frac{A_{\Delta}(P_1, P_2, Q)}{A_\Delta(P_1, P_2, P_3)}$$
wobei $A_\Delta (Q, P_2, P_3) = \frac{1}{2} |P_2 - Q| | P_3 - Q | \sin \phi = \frac{1}{2} | (P_2 - Q) \times (P_3 - Q) |$
      
\newpage
\subsection{Ray Casting - Schnittpunktberechnung}%
\label{rt:sub:ray_casting_schnittpunktberechnung}

\subsubsection{Strahl-Kugel}%
\label{rt:ssub:strahl_kugel}

\begin{itemize}
	\item Implizite Darstellung einer Kugel: $|x - c|^2 - r^2 = 0$ mit Radius r und Mittelpunkt c
	\item Damit lassen sich Werte für $t_{1,2}$ (vgl. Mathematische Aspekte) anhand der Mitternachtsformel berechnen, wenn man die Variablen folgendermaßen wählt
	\begin{itemize}
		\item $a = d * d$
		\item $b = 2d * (e - c)$
		\item $c = (e - c) * (e - c) - r^2$
	\end{itemize}
	\item \textbf{Kein Schnitt}: Diskriminante kleiner 0
	\item \textbf{Kugel wird gestriffen}: Beide t-Werte sind identisch
	\item \textbf{Sonst}: Kugelschnitt, zwei verschiedene t-Werte
	\item \textbf{Hinweis}: Nur $t > 0$ sind relevant!
\end{itemize}

\subsubsection{Strahl-Ebene}%
\label{rt:ssub:strahl_ebene}

\begin{itemize}
	\item Implizite Darstellung einer Ebene: $x * n - d_U = 0$ mit dem Normalenvektor n und dem Abstand vom Ursprung $d_U$
	\item Damit lassen sich Werte für $t$ (vgl. Mathematische Aspekte) anhand der Formel $t = \frac{d_U - e * n}{d * n}$
	\item \textbf{Achtung}: Ist der Nenner $d * n = 0$ sind Strahl und Ebene \textbf{parallel}!
	\item \textbf{Hinweis}: Nur $t > 0$ sind relevant!
\end{itemize}

\subsubsection{Strahl-Dreieck}%
\label{rt:ssub:strahl_dreieck}

\begin{itemize}
	\item Baryzentrische Darstellung eines Punktes in einem Dreieck bestehend aus Punkten $P_1, P_2, P_3: Q = P_1 + \lambda_2(P_2 - P_1) + \lambda_3(P_3 - P_1)$
	\item \textbf{Anmerkung}: $\lambda_1$ ist kein Faktor, die baryzentrischen Koordinaten spannen quasi ein schiefwinkliges Koordinatensystem auf mit dem Ursprung $P_1$
	\item Damit lassen sich Werte für $t$ (vgl. Mathematische Aspekte) durch Lösen der Gleichung $e + t * d = P_1 + \lambda_2(P_2 - P_1) + \lambda_3(P_3 - P_1)$ nach t finden
	\item \textbf{Achtung}: Ist die Gleichung lösbar, so liegt der Schnittpunkt \textbf{in der Ebene des Dreiecks}, damit der Schnittpunkt im Dreieck liegt muss gelten: $\lambda_2, \lambda_3 \geq 0$ und $\lambda_2 + \lambda_3 \leq 1$
	\item \textbf{Hinweis}: Nur $t > 0$ sind relevant!
\end{itemize}

\subsection{Shading}%
\label{rt:sub:shading}

\begin{itemize}
	\item Wörtlich \glqq Schattierung\grqq
	\item Simulation von Oberflächeneigenschaften
	\item Ermöglicht realistische Tiefenwahrnehmung
\end{itemize}

\subsubsection{Materialien}%
\label{rt:ssub:materialien}

\begin{itemize}
	\item Beschreibt Oberflächeneigenschaften
	\item Dadurch: Einfluss auf die Reaktion bei Lichteinfall
	\begin{itemize}
		\item Mattes Material wirkt in erster Linie \textbf{diffus} (keine klare Spiegelung, sehr weich, Licht wird in viele Richtungen gestreut)
		\item Glänzendes/Imperfekt spiegelndes Material besitzt weiche, verschwommene Spiegelungen (\textbf{\glqq glossy\grqq}, zwischen diffus und spekular)
		\item Perfekt spiegelndes Material spiegelt ähnlich wie ein gewöhnlicher Spiegel (\textbf{\glqq specular\grqq}, kaum Streuung)
	\end{itemize}
	\item Reflexionen beschreibt man durch Bidirektionale \textbf{Reflektanzverteilungsfunktionen} (BRDF)
	\begin{itemize}
		\item Generiert durch reale Materialproben sowie Modelle aus Physik und Phänomenologie
		\item Beschreibt Verhältnis von einfallendem zu ausfallendem Licht
		\item Erweiterung auf Transmission (ins Material eindringendes Licht): \textbf{Bidirectional Transmission Distribution Function} (BTDF)
		\item BRDF + BTDF = BSDF (\textbf{Bidirectional Scattering Distribution Function})
	\end{itemize}
\end{itemize}

\subsubsection{Phong-Beleuchtungsmodell}%
\label{rt:ssub:phong_beleuchtungsmodell}

    $$I = k_a I_L + k_d I_L (N \cdot L) + k_s I_L (R_L \cdot V)^n \qquad \text{wobei } R_L = -L + 2N(L \cdot N)$$
\begin{itemize}
	\item Phänomenologisches Modell, modelliert Beleuchtung anhand dreier Komponenten, die aufsummiert den Lichteinfluss ergeben
	\begin{itemize}
		\item \textbf{Ambient}: Grundhelligkeit durch indirekte Beleuchtung: $k_a I_l$
		\item \textbf{Diffus}: Grobe Beleuchtung nach dem Lambertschen Gesetz (beschreibt die Intensitätsabschwächung je nach Material):
                      $k_d I_L (N \cdot L)$
		\item \textbf{Spekular}: Imperfekte Spiegelung, \glqq Highlights\grqq:
                      $k_s I_l (R_L \cdot V)^n$
	\end{itemize}
	\item Ambientes Licht ist meist grundsätzlich vorhanden
	\item Diffuses Licht ergibt sich aus dem diffusen Materialfaktor, der Lichtintensität und dem Punktprodukt von Lichtrichtung und Oberflächennormale
	\item Spekulares Licht ergibt sich aus dem spekularen Materialfaktor, der Lichtintensität und dem Punktprodukt von Lichtreflektionsrichtung und Blickrichtung der Kamera hoch \glqq Phong-Exponent\grqq\ n
	\item Die Lichtintensität \textbf{nimmt mit zunehmender Entfernung zur Lichtquelle ab}!
	\item In der Regel werden die Punktprodukte auf 0 und größer \textbf{\glqq geclampt\grqq}, damit z.B Licht von der Rückseite nicht die Vorderseite beleuchtet
\end{itemize}

\subsection{Schattierung von Dreiecksnetzen}%
\label{rt:sub:schattierung_von_dreiecksnetzen}

\begin{itemize}
	\item Objekt soll kantig erscheinen $\Rightarrow$ jedes Dreieck besitzt \textbf{eine Normale}, die für das Shading des kompletten Dreiecks genutzt wird
	\item Objekt soll glatt erscheinen $\Rightarrow$ \textbf{Interpolation}; berechne gewichtete Summe der Normalen angrenzender Dreiecke für jeden Pixel
\end{itemize}

\subsection{Sekundärstrahlen}%
\label{rt:sub:sekundaerstrahlen}

\subsubsection{Reflexion}%
\label{rt:ssub:reflexion}
    
\begin{itemize}
	\item z.B spiegelnde Metallkugel
	\item Bei Intersektion: Verfolge einen Reflexionsstrahl in die Reflexionsrichtung ab der Position $\epsilon$ Längeneinheiten vor der Intersektion (Vermeidung von erneutem Schneiden derselben Oberfläche) und addiere die resultierende Farbe gewichtet hinzu
\end{itemize}

\textbf{Glanzlichter}\\
Perfekte Spiegelung der Lichtquelle auf Objekt. Fallen mit größerem Phong-Exponenten schneller ab.

\subsubsection{Transmission}%
\label{rt:ssub:transmission}

\begin{itemize}
	\item z.B durchsichtige Glaskugel
	\item Bei Intersektion: Verfolge einen Transmissionsstrahl in die Transmissionsrichtung $\epsilon$ Längeneinheiten von der zweiten Intersektion der Kugel entfernt (Vermeidung von erneutem Schneiden derselben Oberfläche) und addiere die resultierende Farbe gewichtet hinzu
\end{itemize}
\textbf{Brechungsgesetz}\\
Bei Übergang in optisch dichtere Medium wird zum Lot hingebrochen\\
Bei Übergang ins optisch dünnere Medium vom Lot weg\\
$$\eta_i \sin \theta_i = \eta_t \sin \theta_t$$
\textit{Fresnel-Effekt}: Aufspaltung in Reflexion und Transmission; je flacher der Blickwinkel, desto mehr Reflexion und weniger Transmission und andersherum\\
\textit{Totalreflexion}: Bei $\sin \theta_c = \eta_i / \eta_t$ wird gesamtes Licht reflektiert\\
\textit{Disperson}: Ausbreitungsgeschwindigkeit abhängig von Wellenlänge $\rightarrow$ Aufspaltung in Wellenlängen

\subsection{Aliasing}%
\label{rt:sub:aliasing}

\begin{itemize}
	\item Problem: Scharfe, stufige Kanten (\glqq jaggies\grqq), da nur \textbf{grob abgetastet} wird
	\item Lösung: Anti-Aliasing
	\begin{itemize}
		\item Überabtastung (\textbf{Supersampling}, mehrfaches Abtasten desselben Pixels mit leichten Offsets vom Pixelzentrum), dann gewichtete Summe für den Farbwert wählen
		\item Performance-intensiv! Für jeden Pixel müssen nun mehrere Strahlen verfolgt werden, u.U. mit Sekundärstrahlen
	\end{itemize}
\end{itemize}