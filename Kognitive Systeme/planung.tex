\section{Planung}%
\label{plan:sec:planung}

\subsection{Planungssprachen}%
\label{plan:sub:planungssprachen}

\begin{itemize}
	\item \textbf{Ziel}: Zustand der \textbf{Umgebung} sowie \textbf{mögliche Handlungen} repräsentieren
\end{itemize}

\subsubsection{STRIPS}% 
\label{plan:ssub:strips}

\begin{itemize}
	\item \textbf{Zustand}: Konjunktion positiver Literale, Prädikate dürfen \textbf{nur Konstanten} erhalten
	\item \textbf{Geschlossene Welt-Annahme}: Jeder Zustand ist vollständig bekannt, alle nicht angegebenen Literale sind implizit \textbf{falsch}
	\item \textbf{Aktion}: Übergang eines Zustands in einen anderen (\textbf{Zielzustand}), definiert durch \textbf{Name, Parameter, Vorbedingungen und Effekte}
	\item \textbf{Vorbedingung}: Konjunktion positiver Literale
	\item \textbf{Effekte}: Konjunktion positiver und negativer Literale
	\item Ein Zustand \textbf{erfüllt einen Zielzustand}, wenn (mindestens) alle im Zielzustand angegebenen Literale auch in dem Ursprungszustand positiv sind
	\item Eine Aktion ist \textbf{ausführbar}, wenn die Vorbedingungen erfüllt sind; die in den \textbf{Effekten} angegebenen Literale ändern dann ihren Wert entsprechend
	\item \textbf{Beispiel}:
	\begin{itemize}
		\item \texttt{Startzustand}: $Durst\ \land\ IstVolleTasse(t)$
		\item \texttt{Ziel}: $KeinDurst$
		\item \texttt{Aktionen}: $Action(Trinke,\ (t),\ IstVolleTasse(t),\ KeinDurst\ \land\ \neg Durst)$
	\end{itemize}
\end{itemize}

\subsubsection{ADL}% 
\label{plan:ssub:adl}

\begin{itemize}
	\item \textbf{Erweiterung von STRIPS}
	\item \textbf{Offene Welt} im Gegensatz zu STRIPS, Zustände enthalten positive \textbf{und negative} Literale, ungenannte Literale sind \textbf{unbekannt}
	\item Ziele dürfen \textbf{negative Literale} enthalten, Disjunktionen und quantifizierte Variablen sind auch erlaubt
	\item Variablen können auf \textbf{Gleichheit} geprüft werden
	\item \textbf{Implizite Typisierung} ist möglich (\itquote{k ist vom Typ Knopf})
	\begin{align*}
		Action(&Aktiviere(k: Knopf),\ Vorbed:\ Deaktiviert(k),\\ &Effekt:\ Aktiviert(k)\ \land\ \neg\ Deaktiviert(k))
	\end{align*}
\end{itemize}

\newpage
\subsection{Planungsalgorithmen}%
\label{plan:sub:planungsalgorithmen}

\begin{itemize}
	\item \textbf{Ziel}: Ermittlung einer Aktionsfolge, mit der von einem \textbf{Startzustand} zu einem \textbf{Zielzustand} gelangt werden kann
	\item \textbf{Netzwerk von Zuständen} wird hierfür als \textbf{Baum betrachtet}
	\item \textbf{Vorwärtssuche}: Konstruiere rekursiv alle möglichen Zustände vom Start aus, Finde alle Wege die zum Zielzustand führen
	\begin{itemize}
		\item Für große Instanzen \textbf{sehr ineffizient}, besser A* mit guter Heuristik
	\end{itemize}
	\item \textbf{Rückwärtssuche}: Konstruiere rekursiv alle möglichen Zustände vom Ziel aus, Finde alle Wege die zum Startzustand führen
	\begin{itemize}
		\item Besser als Vorwärtssuche, da der Suchraum auf \textbf{relevante Aktionen beschränkt wird}
	\end{itemize}
	\item \textbf{Partial-Order Planner (POP)}: Planungsalgorithmus, welcher \textbf{Reihenfolge} mancher Aktionen \textbf{offen lässt}
\end{itemize}

\subsubsection{A*-Algorithmus}% 
\label{plan:ssub:a_algorithmus}

\begin{itemize}
	\item \textbf{Heuristik}: Funktion zur optimistischen Abstandsschätzung zum Zielknoten ($\leq$ dem echten Abstand)
	\item \textbf{Vorgehen}:
	\begin{enumerate}
		\item Schätze Abstand zum Zielknoten über Heuristik
		\item Wähle Knoten, bei dem \textbf{Schätzwert + bisher dorthin zurückgelegte Strecke} minimal ist
		\item Wiederhole bis das Ziel erreicht wurde
	\end{enumerate}
\end{itemize}

\subsubsection{Planungsgraphen}% 
\label{plan:ssub:planungsgraphen}

\begin{itemize}
	\item \textbf{Ziel}: Graphische Plangewinnung zur Erstellung einer \textbf{Heuristik}
	\item \textbf{Aufbau}:
	\begin{itemize}
		\item Einteilung in \textbf{Abschnitte} für jeden \textbf{Zeitschritt}, sequenzieller Verlauf von links nach rechts
		\item Links von jedem Zeitschritt sind die möglichen \textbf{vorherigen Literalzustände}
		\item Rechts von jedem Zeitschritt sind die möglichen \textbf{resultierenden Literalzustände}
		\item Resultierende Zustände die sich widersprechen werden mit \textbf{Mutex-Link} symbolisiert
		\item Mutex-Links und unveränderte Literale werden in den nächsten Abschnitt \textbf{weitergetragen}
	\end{itemize}
	\item \textbf{Mögliche Heuristikgewinnung}: \textbf{Abschnittszahl} im Planungsgraph als Anzahl notwendiger Schritte interpretieren
	\item \textbf{Mögliche Plangewinnung}: \textbf{Graphplan-Algorithmus}
\end{itemize}