\section{Quantenmechanik}%
\label{quant:sec:quantenmechanik}

\subsection{Historische Erkenntnisse}
\label{quant:sub:historische_erkenntnisse}

\begin{itemize}
	\item \textbf{Hohlraumstrahlung}:
	\begin{itemize}
		\item Elektromagnetische Strahlung im abgeschlossenen Hohlraum im thermischen Gleichgewicht (keine Temperaturänderung durch die Wände)
		\item \textbf{Rayleigh-Jeans-Gesetz} verbindet Lichtwellenlänge und spezifische Ausstrahlung eines Schwarzen Körpers
		\item \textbf{Ultraviolett-Katastrophe}: Rayleigh-Jeans-Gesetz liefert bei kleinen Wellenlängen viel zu große Werte, Versagen der klassischen Physik
	\end{itemize}
	\item \textbf{Welle-Teilchen-Dualismus}:
	\begin{itemize}
		\item Ob Licht sich wie eine Welle oder ein Teilchen verhält hängt vom Experiment ab
		\item \textbf{Licht als Welle}: \textbf{Interferenz} (Wellenüberlagerung)
		\item \textbf{Licht als Teilchen}: \textbf{Photoelektrischer Effekt} (Herausschlagen von Elektronen)
	\end{itemize}
	\item \textbf{Atomphysik}:
	\begin{itemize}
		\item Im Rutherfordschen Atommodell müssten Elektronen ständig Energie abstrahlen und in den Kern stürzen
		\item Emissionssppektrum müsste \textbf{kontinuierlich} sein aufgrund \textbf{kontinuierlich variierender Umlaufbahn}
		\item \textbf{Stattdessen}: Diskrete Emissionslinien, es folgte die \textbf{Quantenhypothese für Elektronenbahnen}
		\item \textbf{Quantenhypothese}: Energiemenge die Strahlung und Materie austauschen können ist \textbf{nicht beliebig}
	\end{itemize}
	\item \textbf{Teilchenwellen}: Welle-Teilchen-Dualismus gilt auch für \textbf{konventionelle Teilchen} (Elektronen)
\end{itemize}

\subsection{Schrödinger Gleichung}
\label{quant:sub:schroedinger_gleichung}

\begin{itemize}
	\item Beschreibt in Form einer partiellen Differentialgleichung die \textbf{zeitliche Veränderung des quantenmechanischen Zustands} eines \textbf{nichtrelativistischen} Systems
	\item \textbf{Zeitabhängige Schrödingergleichung}:
	$$
		ih\frac{\partial}{\partial t}\psi(\vec{x},t) = H\psi(\vec{x}, t) = \frac{-h^2}{2m}\Delta\psi(\vec{x}, t)
	$$
	\item \textbf{Zeitunabhängige Schrödingergleichung}:
	$$
		H\phi(\vec{r}) = E\phi(\vec{r})
	$$
\end{itemize}