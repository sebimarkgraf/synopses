\section{Lagrangeformalismus}%
\label{lag:sec:lagrangeformalismus}

\begin{itemize}
	\item Umformulierung der Newtonschen Mechanik
	\item \textbf{Lagrangefunktion}: Skalare Funktion, umfasst gesamte klassische Mechanik
	\item \textbf{Vorteil}: Einfache Behandlung von Problemen mit Zwangsbedingungen (z.B Bewegung von verbundenen Kugeln)
\end{itemize}

\subsection{Lagrangegleichungen 1. Art}%
\label{lag:sub:lagrangegleichungen_1_art}

\begin{itemize}
	\item \textbf{Gegeben}: System aus $N$ Massenpunkten mit Massen $m_i, i = 1..N$
	\item \textbf{Freiheitsgrade}: Massenpunkte können sich in z.B $3$ Dimensionen bewegen $\Rightarrow$ $3N$ Freiheitsgrade
	\item \textbf{Holonome Zwangsbedingungen}: $N_Z$ Zwangsbedingungen von $N$ Teilchen mit Koordinaten $\vec{r_1}..\vec{r_N}$ sind \textbf{holonom}, wenn sie sich folgendermaßen schreiben lassen: $$A_\mu(\vec{r_1}..\vec{r_N}, t) = 0,\ \mu = 1..N_Z$$ ansonsten sind sie \textbf{nichtholonom} (z.B Ungleichungen)
	\item \textbf{Skleronome \& Rheonome Zwangsbedingungen}: Zwangsbedingungen mit expliziter Zeitabhängigkeit sind \textbf{rheonom}, alle anderen \textbf{skleronom}
\end{itemize}

\subsection{Lagrangegleichungen 2. Art}%
\label{lag:sub:lagrangegleichungen_2_art}

\subsection{Erhaltungsgrößen}%
\label{lag:sub:erhaltungsgroessen}
