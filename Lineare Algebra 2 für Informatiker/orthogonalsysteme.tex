\section{Orthogonalsysteme}%
\label{os:sec:orthogonalsysteme}

Sei K ein Körper und V ein K-Vektorraum mit Basis $B = \{b_1, ..., b_n\}$ sowie einem Skalarprodukt $\langle \cdot, \cdot\rangle$ und einer Norm $|\cdot|$.

\subsection{Gram-Schmidt Orthogonalisierung und -normalisierung}%
\label{os:sub:gram_schmidt_orthogonalisierung_und_normalisierung}

Ziel: Ermittlung einer Basis aus orthogonalen Vektoren bzw. orthonormalen Vektoren (Vektoren haben Länge 1, wie z.B die Standardbasisvektoren bzgl. der Standardmetrik).\\\\
\textbf{Vorgehen}:
\begin{enumerate}
	\item Wähle Startvektor $b^*_1 = b_1$ (und normiere diesen für eine Orthonormalbasis mit $c_1 =  \frac{b^*_1}{\norm{b^*_1}}$)
	\item Iteriere durch die restlichen Basisvektoren und berechne
	\begin{enumerate}
		\item Abziehen der parallelen Teile der vorherigen Basisvektoren, damit der neue Vektor\\orthogonal zum Rest ist: $$b^*_i = b_i - \sum_{j = 1}^{i - 1} \frac{\langle b_i, b^*_j\rangle}{\langle b^*_j, b^*_j\rangle}b^*_j$$\vspace*{-0.6cm}
		\item Normierung: $c_i = \frac{b^*_i}{\norm{b^*_i}}$
	\end{enumerate}
\end{enumerate}
Damit ergibt sich die Orthogonalbasis bzgl. des Skalarprodukts und der Norm: $B^* = \{b^*_1, ..., b^*_n\}$ sowie (optional) die Orthonormalbasis $C = \{c_1, ..., c_n\}$.

\subsection{Orthogonale Komplemente und Projektionen}%
\label{os:sub:orthogonale_komplemente_und_projektionen}

\textbf{Allgemein:}
\begin{itemize}
	\item Das \textbf{orthogonale Komplement} einer Teilmenge M des Vektorraums V entspricht:
	\begin{center}
		$M^\perp := \{v \in V\ |\ v\perp m\ \forall m\in M\} = \{v\in V\ |\ \langle v, m\rangle = 0\ \forall m\in M\}$
	\end{center}
	\item Es gilt für einen Untervektorraum U zu V:
	\begin{center}
		$V = U \bigoplus U^\perp$ und $dim V = dim U + dim U^\perp$
	\end{center}
	\item Weiter gibt es die \textbf{orthogonale Projektion}, die das orthogonale Komplement unterschlägt:
	\begin{center}
		$\Pi_U: V = U \bigoplus U^\perp \rightarrow U, v = u + u^\perp \mapsto u$
	\end{center}
	wobei $ker \Pi_U = U^\perp,\ \Pi_{U|U} = id_U$ und $\Pi^2_U = \Pi_U$
	\item $\Pi_U(v)$ ist der \textbf{Lotfußpunkt} von v auf U und hat in U den kürzesten Abstand zu v
\end{itemize}

\newpage
\subsection{Berechnung der orthogonalen Projektion}%
\label{os:sub:berechnung_der_orthogonalen_projektion}	
\begin{enumerate}
	\item Berechne die \textbf{Orthonormalbasis} $\{c_1, ..., c_m\}$ \textbf{von U}
	\item Ergänze diese zu einer \textbf{Orthonormalbasis von V}, d.h. $\{c_1, ..., c_m, c_{m + 1}, ... c_n\}$, der ergänzte Teil ist die Orthonormalbasis zu $U^\perp$
	\item Damit folgt die orth. Projektion: $\Pi_U: V \rightarrow U, b_i \mapsto
	\begin{cases}
		b_i,& falls\ i \in \{1, ..., m\}\\
		0,& falls\ i \in \{m + 1, ..., n\}\\
	\end{cases}$
\end{enumerate}
Für die \textbf{direkte Berechnung} der Orthogonalprojektion eines Vektors v aus V auf U gilt mit der Orthonormalbasis $\{c_1, ..., c_m\}$ von U:
\begin{center}
	$\Pi_U(v) = \sum_{i = 1}^{n} \langle v, c_i \rangle c_i$
\end{center}
\textbf{Berechnung des Abstands}:
\begin{itemize}
	\item Für den Abstand eines Vektors v zum Untervektorraum U gilt:
	\begin{center}
		$d(v, U) = d(v, \Pi_U(v)) = ||v - \Pi_U(v)||$
	\end{center}
	\item Damit: Lotfußpunkt ausrechnen und Abstand per gegebener Norm ausrechnen
\end{itemize}

\subsection{Orthogonale und Unitäre Gruppen}%
\label{os:sub:orthogonale_und_unitaere_gruppen}

\begin{itemize}
	\item \textbf{Orthogonale Gruppe}: $O(n) := \{A \in \realnumbers^{N\times N}\ |\ A^TA = I_N\}$
	\item \textbf{Spezielle orthogonale Gruppe}: $SO(n) := \{A \in O(n)\ |\ det A = 1\}$
	\item \textbf{Unitäre Gruppe}: $U(n) := \{A \in \complexnumbers^{N\times N}\ |\ \overline{A}^TA = I_N\}$
	\item \textbf{Spezielle unitäre Gruppe}: $SU(n) := \{A \in U(n)\ |\ det A = 1\}$
\end{itemize}

\subsection{Iwasawa-Zerlegung}%
\label{os:sub:iwasawa_zerlegung}

\textbf{Ziel}: Zerlegen einer Matrix $A \in \realnumbers^{N\times N} (\complexnumbers^{N\times N})$ in das Produkt einer orthogonalen (unitären) Matrix O und einer oberen Dreiecksmatrix R mit positiven reellen Diagonaleinträgen, sodass: $A = OR$.\\\\
\textbf{Vorgehen}:
\begin{enumerate}
	\item Bilde eine \textbf{Orthonormalbasis} $C = \{c_1, ..., c_n\}$ aus den Basisvektoren in den Spalten der Matrix A ($A = (a_1 | a_2 | ... | a_n)$)
	\item O entspricht den \textbf{Basisvektoren der Orthonormalbasis} $c_i$, d.h. $O = (c_1 | c_2 | ... | c_n)$
	\item Bilde R über \textbf{Gauß-Jordan}, d.h. entwickle A nach O, d.h. löse das LGS $(O | A)$, sodass die linke Matrix $I_N$ entspricht; die \textbf{rechte Matrix entspricht dann R}
\end{enumerate}