\section{Wellen}%
\label{well:sec:wellen}

\subsection{Wellenausbreitung und Wellengleichung}%
\label{well:sub:wellenausbreitung_und_wellengleichung}

\begin{itemize}
	\item \textbf{Welle}: Sich räumlich ausbreitende Schwingung
	\item \textbf{Frequenz} (Einheit: \textbf{Herz [Hz]}, mit Periodendauer $T$):
	\begin{equation}
		f = \frac{Schwingungen}{\Delta Zeit} = \frac{1}{T}
	\end{equation}
	\item \textbf{Kreisfrequenz}:
	\begin{equation}
		\omega = 2\pi f
	\end{equation}
	\item \textbf{Wellenzahl} (mit Wellenlänge $\lambda$)
	\begin{equation}
		k = \frac{2\pi}{\lambda}
	\end{equation}
	\item \textbf{Phasengeschwindigkeit}:
	\begin{equation}
		v_{ph} = \frac{\omega}{k}
	\end{equation}
	\item \textbf{Gruppengeschwindigkeit}:
	\begin{equation}
		v_g = \frac{d\omega}{dk}
	\end{equation}
	\item \textbf{Longitudinalwelle}: Bewegung senkrecht zur Ausbreitungsrichtung
	\item \textbf{Transversalwelle}: Bewegung entlang der Ausbreitungsrichtung
	\item \textbf{Wellenausbreitung}:
	\begin{equation}
		y(x, t) = Asin(\omega t - kx)
	\end{equation}
	\item \textbf{Wellengleichung für elektromagnetische Wellen}:
	\begin{equation}
		\vec{\nabla}^2\vec{B} - \mu_0\epsilon_0\partial_t^2\vec{B} = 0
	\end{equation}
	\item \textbf{Dispersionsrelation}:
	\begin{equation}
		\omega(k) = \sqrt{\frac{1}{\epsilon_0\mu_0}}k
	\end{equation}
\end{itemize}

\subsection{Interferenz und Beugung}%
\label{well:sub:interferenz_und_beugung}

\begin{itemize}
	\item \textbf{Superposition (Interferenz)}: Auslenkungen von Teilwellen addieren sich linear zur Auslenkung der Gesamtwelle
	\item \textbf{Stehende Wellen} (manche Stellen sind konstant $0$):
	\begin{equation}
		y = 2Asin(\omega t)cos(kx)
	\end{equation}
	\item \textbf{Schwebung}: Superposition zweier Schwingungen mit nahezu identischer Frequenz zeigt periodisch zu- und abnehmende Amplitude
	\item \textbf{Huygens'sche Elementarwellen}: \itquote{Jeder Punkt einer Welle ist Ausgangspunkt einer neuen Kugelwelle.} Frequenz bleibt gleich, Ausbreitungsgeschwindigkeit und Wellenlänge abhängig vom Medium; Beugung, Reflexion, Brechung etc. sind die Konsequenz
	\item \textbf{Beugung}:
	\begin{itemize}
		\item \textbf{Licht an einer endlichen Kante} dringt in den Schattenbereich hinter der Kante ein
		\item \textbf{Licht am Einzelspalt} breitet sich ringförmig im Schattenbereich aus
	\end{itemize}
	\item \textbf{Reflexion}:
	\begin{itemize}
		\item \textbf{Einlaufende Welle}:
		\begin{equation}
			y = Asin(\omega t - kx)
		\end{equation}
		\item \textbf{Auslaufende Welle} bei einem \textbf{festen Ende} (z.B Schnur an Wand befestigt, anderes Ende wird geschwungen; \textbf{Phasensprung}, Wellenberg wird -tal und andersherum und Welle läuft rückwärts zurück):
		\begin{equation}
			y = Asin(\omega t + kx + \pi)
		\end{equation}
		\item \textbf{Auslaufende Welle} bei einem \textbf{losen Ende} (z.B lose Schnur schwingen; \textbf{kein Phasensprung}, Welle läuft in gleicher Form wieder zurück):
		\begin{equation}
			y = Asin(\omega t + kx)
		\end{equation}
		\item \textbf{Reflexionsgesetz}: $Einfallswinkel = Ausfallswinkel$
	\end{itemize}
	\item \textbf{Brechung}: Eindringen in Medium mit ggf. Änderung des Winkels; \textbf{Snellsches Brechungsgesetz}:
	\begin{equation}
		\frac{sin\phi_1}{sin\phi_2} = \frac{v_1}{v_2} = \frac{\lambda_1}{\lambda_2} = \frac{n_2}{n_1}
	\end{equation}
	\item \textbf{Totalreflexion}: Gebrochener Strahl parallel zur Übergangsgerade der Medien
	\item \textbf{Dispersion}: Wellenlängenbedingte unterschiedliche Brechung von Licht an einem Medium
	\item \textbf{Doppelspalt}: Einfallendes Licht trifft durch zwei Spalte hervor und interferiert; Interferenzmuster werden auf einem Schirm sichtbar (Maximum in der Mitte, nach außen abwechselnd Minima und Maximima abfallender Intensität; Spaltabstand sei $d$)
	\begin{itemize}
		\item \textbf{n-tes Maximum}:
		\begin{equation}
			sin\phi_{max} = n\frac{\lambda}{d}, n = 0, 1, \dots
		\end{equation}
		\item \textbf{n-tes Minimum}
		\begin{equation}
			sin\phi_{min} = (n + \frac{1}{2})\frac{\lambda}{d}, n = 0, 1, \dots
		\end{equation}		
	\end{itemize}
	\item \textbf{Gitter}: Analog zu Doppelspalt, nur mit größerer Spaltzahl; erzeugt höhere und schärfere Beugungsmaxima; Formeln analog nur mit Gitterabstand statt Spaltabstand
	\item \textbf{Polarisation}: \quotestyle{Orientierung der Welle} (z.B Transversalwellen sind horizontal bzw. vertikal polarisiert); kann durch \textbf{Polarisationsfilter} angepasst werden, reduziert aber Intensität (alternativ: Polarisation durch Reflexion)
\end{itemize}

\subsection{Geometrische Optik}%
\label{well:sub:geometrische_optik}

\begin{itemize}
	\item \textbf{Strahlenmodell}: Licht als Strahl, vernachlässige Welleneigenschaften wie z.B Beugung; ausreichend, sofern alle relevanten Längen $<< \lambda$
	\item \textbf{Konvex}: Nach außen gewölbt
	\item \textbf{Konkav}: Nach innen gewölbt
	\item \textbf{Hohlspiegel}: Konkaver Spiegel; sieht aus wie ein Kugelausschnitt
	\item \textbf{Brennpunkt}: $F$ = Sammelpunkt aller parallelen Lichtstrahlen in einer optischen Linse bzw. einem Hohlspiegel
	\item \textbf{Brennweite}: $f$ = Halber Radius im Falle eines Hohlspiegels
	\item \textbf{Gegenstandsweite}: $d_O$ = Abstand zwischen Spiegel und Objekt
	\item \textbf{Bildweite}: $d_B$ = Abstand zwischen Spiegel und Bild
	\item \textbf{Sammellinse (Konvexlinse)}: Paralleles Licht läuft hinter der Linse zusammen; Linsengleichung:
	\begin{equation}
		\frac{1}{d_O} + \frac{1}{d_B} = \frac{1}{f}		
	\end{equation}
	\item \textbf{Zerstreuungslinse (Konkavlinse)}: Paralleles Licht läuft hinter der Linse auseinander; Linsengleichung:
	\begin{equation}
		\frac{1}{d_O} - \frac{1}{d_B} = -\frac{1}{f}
	\end{equation}
\end{itemize}