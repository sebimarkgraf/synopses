\section{Schlüsselaustauschprotokolle}%
\label{schaus:sec:schluesselaustauschprotokolle}

\begin{itemize}
	\item \textbf{Ziel}: Sicherer Austausch von Schlüsseln über eine unsichere Leitung
	\item \textbf{Schlüsselzentrale}: Jeder Teilnehmer hat die Möglichkeit zur sicheren Kommunikation mit einem zentralen dritten Partner
	\item \textbf{Symmetrische Verfahren}: Es sei angenommen, beide Partner können mit der Schlüsselzentrale kommunizieren
	\begin{itemize}
		\item \textbf{Kerberos}:
		\begin{itemize}
			\item A stellt Anfrage an Schlüsselzentrale, erhält zwei verschlüsselte Pakete mit dem Sitzungsschlüssel und dessen Lebensdauer
			\item A öffnet das für A verschlüsselte Paket und sendet das für B verschlüsselte Paket an B
			\item A sendet eine weitere Nachricht mit der eigenen Identität und einem Zeitstempel an B
			\item A und B können nun sicher kommunizieren
		\end{itemize}
	\end{itemize}
	\item \textbf{Asymmetrische Verfahren}: Es sei angenommen, dass jeder Teilnehmer ein eigenes Schlüsselpaar hat und die öffentlichen Schlüssel an einer vertrauenswürdigen Stelle hinterlegt sind
	\begin{itemize}
		\item \textbf{Public-Key Transport}: A erzeugt einen Sitzungsschlüssel, chiffriert diesen mit dem öffentlichen Schlüssel von B und versendet ihn (ggf. auch mit Signaturen)
		\item \textbf{Diffie-Hellman-Schlüsselaustausch}:
		\begin{itemize}
			\item Ursprung der ElGamal-Verschlüsselung
			\item A und B wählen ganze Zufallszahlen $x, y$ und senden sich gegenseitig $g^x$ bzw. $g^y$, $g$ ist das Erzeugerelement der zyklischen Gruppe $G$; beide können nun das gemeinsame Geheimnis $g^{xy}$ berechnen, Angreifer jedoch nicht, da diese nur $g$, $g^x$ und $g^y$ kennt
			\item \textbf{Man-in-the-Middle-Angriff} ist möglich, wenn $x, y$ jeweils abgefangen und ersetzt werden; dagegen helfen \textbf{Signaturen}
		\end{itemize}
	\end{itemize}
	\item \textbf{Transport Layer Security (TLS)}:
	\begin{itemize}
		\item Weiterentwicklung von \textbf{SSL (Secure Socket Layer)}; hier vor allem die \textbf{Handshake-Phase} interessant, die den Schlüsselaustausch durchführt
		\begin{enumerate}
			\item Client schickt Server Zufallszahl und nach Präferenz sortierte Liste von Algorithmen (Hashfunktionen, Verschlüsselungen, Schlüsselaustauschprotokolle)
			\item Server sendet seine Wahl sowie ein Zertifikat mit seinem öffentlichen Schlüssel zurück, fordert nun das Zertifikat des Clients
			\item Client antwortet mit seinem Zertifikat und seinem öffentlichen Schlüssel sowie dem signierten Hashwert der bisherigen Nachrichten
			\item Client wählt weitere Zufallszahl (\textbf{Premaster Secret (PMS)}), sendet diesen an den Server, beide berechnen aus den Zufallszahlen den \textbf{Master Key (MS)}
			\item Client signalisiert, dass nun verschlüsselt kommuniziert wird und beendet den Handshake, Server antwortet analog
			\item Kommunikation läuft fortan über den \textbf{Session Key}, welcher aus dem \textbf{Master Key} pseudozufällig abgeleitet wird
		\end{enumerate}
		\item TLS ist mit idealer Verschlüsselung \textbf{sicher}, Angriffe meist durch Implementierung bedingt
		\item \textbf{ChangeCipherSpec-Drop-Angriff}: Unterdrücke vorletzten Schritt, Server kommuniziert noch unverschlüsselt, Nutzdaten werden ausgelesen
		\item \textbf{RSA-Padding-Angriff}: RSA mit \quotestyle{naivem} Padding und veraltete SSL-Versionen ermöglichen Einblick in den Session Key
		\item \textbf{CRIME-Angriff}: Angreifer kann sich bei aktivierter Kompression Stück für Stück dem Klartext nähern
		\item \textbf{Logjam-Angriff}: Angreifer überzeugt per Man-in-the-Middle den Server, ein unsicheres Verfahren zu verwenden
	\end{itemize}
	\item \textbf{Internet Protocol Security (IPsec)}: Protokoll zur Integritätssicherung, Erkennung falscher IP-Adressen (IP-Spoofing) und Verschlüsselung von IP-Paketen
	\item \textbf{Password Authentication Key Exchange (PAKE)}:
	\begin{itemize}
		\item \textbf{Grundlegendes Prinzip, kein Protokoll}
		\item A und B besitzen bereits ein gemeinsames Passwort, über das sie kommunizieren wollen
		\item Sollte ein Angreifer das Passwort knacken, sollte bisherige Kommunikation nicht mehr nachvollziehbar sein (\textbf{forward-secrecy})
		\item \textbf{Funktionierende Konstruktion: Encrypted Key Exchange (EKE)} startet Kommunikation mit verschlüsseltem öffentlichem Schlüssel, danach werden asymmetrische Verfahren genutzt
		\item \textbf{Basis von EAP und WPA}
	\end{itemize}
\end{itemize}