\section{Asymmetrische Verschlüsselung}%
\label{asver:sec:asymmetrische_verschluesselung}

\begin{itemize}
	\item \textbf{Idee}: Ein Schlüssel zur Verschlüsselung (\quotestyle{public key}, von jedem einsehbar), ein Schlüssel zur Entschlüsselung (\quotestyle{private key}, wird geheim gehalten)
	\item \textbf{Wichtig}:
	\begin{itemize}
		\item Es muss schwer sein, vom öffentlichen Schlüssel auf den privaten zu schließen
		\item Der öffentliche Schlüssel eines anderen darf vor dem Erhalt nicht durch Dritte manipuliert werden
	\end{itemize}
	\item \textbf{Erweiterter Euklidischer Algorithmus (EEA)}: Bestimmt das inverse Element zu einem Element einer multiplikativen Gruppe. Für gegebene Parameter $A$, $B$ werden $ggT(A, B)$ sowie ganze Zahlen $s$, $t$ berechnet, sodass gilt:\\
	\begin{align*}
		ggT(A, B) = s \cdot A + t \cdot B\text{.}
	\end{align*}
\end{itemize}

\subsection{RSA}%
\label{asver:sub:rsa}

\begin{itemize}
	\item \textbf{Bestandteile}:
	\begin{enumerate}
		\item \textbf{Generator-Algorithmus}:
		\begin{enumerate}
			\item Wähle zwei große Primzahlen $P, Q, P \neq Q$ mit Bitlänge $k$
			\item Berechne: $N = P \cdot Q$ und $\phi(N) = (P - 1)(Q - 1)$
			\item Wähle $e \in \{3, \dots, \phi(N) - 1\}, ggT(e, \phi(N)) = 1$
			\item Berechne das zu e multiplikativ-inverse Element $d$ per EEA
			\item \textbf{Privater Schlüssel}: $sk = (N, d)$, \textbf{Öffentlicher Schlüssel}: $pk = (N, e)$
		\end{enumerate}
		\item \textbf{Ver- und Entschlüsselungsalgorithmus}:
		\begin{align*}
			ENC(pk, M) &= M^e\ mod\ N\\
			DEC(sk, C) &= C^d\ mod\ N
		\end{align*}
	\end{enumerate}
	\item \textbf{Sicherheit}:
	\begin{itemize}
		\item Bisher gezeigter RSA, auch \textbf{Textbook-RSA}, weist einige Schwächen auf
		\item \textbf{Probleme von Textbook-RSA}:
		\begin{enumerate}
			\item RSA ist \textbf{deterministisch}, Nachricht wird mit demselben Schlüssel immer zur selben Chiffre
			\item Löst man das Problem der \textbf{Faktorisierung von natürlichen Zahlen in Polynomialzeit} ist RSA \textbf{gebrochen}
			\item Weitere Angriffsmöglichkeiten ergeben sich, wählen mehrere Benutzer dasselbe e \textbf{oder} N
		\end{enumerate}
	\end{itemize}
	\item \textbf{Sicheres RSA (RSA-OAEP)}:
	\begin{itemize}
		\item \quotestyle{RSA optimal asymmetric encryption padding} ist IND-CCA-sicher
		\item \textbf{Verschlüsselung}: Zufallszahl wandelt Nachricht vor Verschlüsselung ab (\textbf{Padding}); ist rekonstruierbar, kann also danach verworfen werden
		\item \textbf{Entschlüsselung}: Entschlüssele Nachricht, rekonstruiere Zufallszahl, mache Padding rückgängig
	\end{itemize}
	\item \textbf{Bedeutung von RSA}: Muss angepasst werden, um konkurrenzfähige Laufzeiten zu erhalten, danach aber in der Praxis oft anzutreffen
\end{itemize}

\subsection{ElGamal}%
\label{asver:sub:elgamal}

\begin{itemize}
	\item \textbf{Idee}: Basiere Verschlüsselung auf DLOG-Problem (Berechnung diskreter Logarithmen in zyklischen Gruppen)
	\item \textbf{Zyklische Gruppen}: Gruppe $G$ mit Erzeugerelement $g$, sodass jedes Gruppenelement eine (ganzzahlige) Potenz des Erzeugerelements ist. Per Definition gilt Ordnung $q = \frac{(p - 1)}{2}$
	\item \textbf{Schlüsselerzeugung}: Wähle $x \in \{2, \dots, p - 1\}, h \equiv g^x$, damit folgt dann:
	\begin{align*}
		pk &= (G, g, h)\\
		sk &= (G, g, x)
	\end{align*}
	\item \textbf{Ver- und Entschlüsselung}: Wähle zur Verschlüsselung $y \in \{2, \dots, p - 1\}$ zufällig, dann gilt:
	\begin{align*}
		ENC(pk, M) &= (g^y, h^yM)\\
		DEC(sk, (g^y, C)) &= \frac{C}{(g^y)^x}
	\end{align*}
	\item \textbf{Sicherheit}:
	\begin{itemize}
		\item Wahl von $G$ entscheidend, ist nur IND-CPA-sicher, wenn in $G$ die DDH-Annahme gilt; $G$ benötigt hierzu ausreichend Elemente
		\item \textbf{Decisional Diffie-Hellman-Annahme}: In einer Gruppe $G$ sind die Tupel $(g^a, g^b, g^{ab})$ und $(g^a, g^b, g^c)$ für zufällig und unabhängig gewählte $a, b, c$ von jedem PPT-Angreifer nur mit im Sicherheitsparameter $k$ vernachlässigbarer Wahrscheinlichkeit unterscheidbar
	\end{itemize}
	\item \textbf{Problem}: Nur Nachrichten, die in $G$ liegen, können verschlüsselt werden; \textbf{Lösungsansätze} sind wie folgt:
	\begin{itemize}
		\item Verfahren zur \textbf{Nachrichtenumwandlung} übersetzen zwischen Binärdarstellung und einer Nachricht in $G$, ist aber ineffizient
		\item Das \textbf{Hash-ElGamal-Kryptosystem} verwendet ElGamal mit einer Hashfunktion; dies vergrößert den Urbildraum und ist \textbf{nicht identisch mit Nachrichtenumwandlung}, da nicht auf $G$ abgebildet wird
	\end{itemize}
\end{itemize}
