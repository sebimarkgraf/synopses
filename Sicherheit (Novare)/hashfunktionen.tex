\section{Hashfunktionen}%
\label{hash:sec:hashfunktionen}

\begin{itemize}
	\item \textbf{Hashfunktionen} bilden eine große, potentiell unbeschränkte Menge in eine kleinere Menge ab. Mit Sicherheitsparameter $k$:
	$$
		H_k: \{0, 1\}^* \rightarrow \{0, 1\}^k
	$$
	\item \textbf{Sicherheitseigenschaften}:
	\begin{itemize}
		\item \textbf{Collision Resistance}: Zwei Urbilder mit demselben Hashwert zu finden sollte möglichst schwierig sein. Formal:
		\begin{align*}
			Adv^{cr}_{H,A}(k):= Pr \left[ (X, X') \leftarrow A(1^k) : X \neq X'\land H_k(X) = H_k(X')\right]
		\end{align*}
		für PPT-Angreifer $A$ in $k$ vernachlässigbar.
		\item \textbf{Pre-Image Resistance}: Hashfunktion darf nur in eine Richtung berechenbar sein (Urbild finden ist möglichst schwierig). Formal:
		\begin{align*}
			Adv^{ow}_{H,A}(k):= Pr \left[ X' \leftarrow A(H(X), 1^k) : H(X) = H(X')\right]
		\end{align*}
		für PPT-Angreifer $A$ in $k$ vernachlässigbar.
		\item \textbf{Target Collision Resistance}:
		\begin{itemize}
			\item Zwischenschritt zwischen Collision Resistance und Pre-Image Resistance
			\item Es muss schwierig sein, für ein \textbf{gegebenes} Urbild $X$ ein $X' \neq X$ zu finden, für das gilt: $H(X') = H(X)$
			\item Formal: $H$ genügt Target Collision Resistance, falls
			\begin{align*}
				Adv^{tcr}_{H,A}(k):= Pr \left[ X' \leftarrow A(X, 1^k) : X \neq X' \land H_k(X) = H_k(X')\right] 
			\end{align*}
			für PPT-Angreifer $A$ in $k$ vernachlässigbar.
		\end{itemize}
		\item Collision Resistance $\Rightarrow$ Target Collision Resistance $\Rightarrow$ Pre-Image Resistance
	\end{itemize}
	\item \textbf{Merkle-Damgard-Transformation}:
	\begin{itemize}
		\item \textbf{Ziel}: Hashfunktion, die obige Eigenschaften einhält und aus Eingaben beliebiger Länge Ausgaben konstanter Länge erzeugt
		\item \textbf{Struktur}: 
		\begin{enumerate}
			\item Aufteilung einer Nachricht in Blöcke $M_1, \dots, M_n$ fester Länge $l$
			\item Iterative Kompression der einzelnen Blöcke mit dem Initialisierungsvektor $IV$ und der Kompressionsfunktion $F$ zu Bitströmen $Z_1, \dots, Z_n$ fester Länge $k \leq l$. $Z_n$ ist der Hashwert:
			\begin{align*}
				Z_0 &= IV\\
				\forall i \in \{1, \dots, n\}: Z_i &= F(Z_{i-1}\ ||\ M_i)
			\end{align*}
		\end{enumerate}
	\end{itemize}
	\item \textbf{Secure Hash Algorithm (SHA)}:
	\begin{itemize}
		\item \textbf{Theoretisch gebrochen}, aber immer noch verbreitet (z.B für Prüfsummen), basiert auf Merkle-Damgard-Konstruktion
		\item \textbf{MD5}: Alternative Hashfunktion, ist aber auch gebrochen, mindestens SHA-2 ist zu empfehlen (verwundbar, aber wegen größerem Bildraum sicherer)
	\end{itemize}
	\item \textbf{Angriffsmöglichkeiten}:
	\begin{itemize}
		\item \textbf{Ziel}: Andere Daten finden, die denselben Hashwert ergeben (Brechen der Datenintegrität)
		\item \textbf{Birthday-Attack}: Berechne viele Hashwerte aus gleichverteilten Urbildern, ordne in sortierte Liste und suche nach Kollisionen
		\item \textbf{Meet-in-the-Middle}: Funktioniert, wenn Hashfunktion eine \quotestyle{Rückwärtsberechnung} zulässt
	\end{itemize}
\end{itemize}