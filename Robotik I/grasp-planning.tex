
%%% Local Variables:
%%% mode: latex
%%% TeX-master: "robotik"
%%% End:

\section{Greifplanung}

\subsection{Grifftaxonomie}
\begin{itemize}
\item Vereinfachung der Griffsynthese
\item Benchmark für die Evaluierung von Roboterhänden
\item Grundlagen für das Design von Roboterhänden
\item Einsatz bei autonomer Greifplanung
\end{itemize}

\textbf{Cutkosky Grifftaxanomie}\\
16 Grifftypen, unterschieden in Form des Objekts und Fingerstellung\\
Hierarchisch\\
Oberste Unterscheidung: Kraft und Präzisionsgriffe

\subsection{Greifanalyse und Greifsynthese}
\textbf{Griff}\\
Eine Menge von Kontaktpunkten auf der Oberfläche eines Objekts, die potentielle Bewegungen des Objekts unter dem
Einfluss externer Kräfte einschränken bzw. kompensieren\\

\textbf{Greifanalyse}\\
Gegeben: Objekt und ein Griff als Menge von Kontaktpunkten\\
Gesucht: Aussagen zur Stabilität des Griffs unter Berücksichtigung von Nebenbedingungen\\

\textbf{Greifsynthese}\\
Gegeben: Objekt und eine Menge von Nebenbedingungen\\
Gesucht: Eine Menge von Kontaktpunkten

\subsection{Griffmodelle}
\subsubsection{Wrenchvektor}
In Kontaktpunkt \(p_i\) wirkenden Kräfte \(f_i\) und Momente \(\tau_i\) mit \(i \in \{x, y, z\}\) in einem Vektor,
genannt Wrenchvektor, \(w\) zusammengefasst.

\subsubsection{Greifmatrix}
Darstellung eines räumlichen Griffes durch Wrenchvektoren mit \(m\) Kontaktpunkten
\[G = \left[ {}^1w_n, {}^1w_t, {}^1w_{\theta}, \ldots, {}^mw_n, {}^mw_t, {}^mw_{\theta}\right] \in \realnumbers^{6\times3m}\]

\subsubsection{Fingerspitzengriff}
Greifen nur durch geeignete Anordnung der Kontaktpunkte\\
Arten der Kontakte:
\begin{itemize}
\item Punktkontakt ohne Reibung --- Kraft normal zur Oberfläche
\item Starrer Punktkontakt mit Reibung --- Kraft normal und tangential zur Fläche
\item Nicht starrer Punktkontakt mit Reibung --- Kraft normal und tangential zur Fläche, sowie axiale Momente
\end{itemize}

\subsubsection{Gleichgewichtsgriff}
Griff ist ein Gleichgewichtsgriff, wenn die Summe aller Kräfte \(f_i\) und Momente \(\tau_i\), die auf das gegriffene
Objekt wirken, gleich Null ist.

\subsubsection{Kraftgeschlossene Griffe}
Ausgleichen von nicht bekannten externen Einflüssen durch Kräfte der Finger.\\
mathematisch
\begin{align}
  \forall e = (f_x, f_y, f_z, \tau_x, \tau_y, \tau_z)^T \in \realnumbers^6 \\
  \exists c \in \realnumbers^{3m}, \quad c \neq 0: \quad G \cdot c + e = 0
\end{align}

\subsubsection{Formgeschlossene Griffe}
Stärkere Einschränkungen als kraftgeschlossender Griff\\
Nur von der Position der Kontaktpunkte auf der Oberfläche abhängig\\
Keine Berücksichtigung von Normal-/Tangentialkräften oder Drehmomenten

\subsubsection{Stabile Griffe}
Nachgiebigkeit von Fingern und Kräften mit Hilfe von Potentialfunktion \(V\)
\[V : \realnumbers^6 \rightarrow \realnumbers^6\]
Potentielle Energie in Abhängigkeit von Lage und Orientierung des Objektes

Griff stabil, falls für kleine Änderungen \(\delta q = (\delta_x, \delta_y, \delta_z, \delta_\alpha,\delta_\beta, \delta_\gamma) \in
\realnumbers^6 \neq 0\) der Lage
\[\forall \delta q \in \realnumbers^6: \quad \delta V > 0\]


\subsection{Objektklassen}
\begin{itemize}
\item Bekannte Objekte
\item Bekannte Objektklassen
\item Unbekannte Objekte
\end{itemize}

\subsection{Griffsynthese}
\subsubsection{Durch Vorwärtsplanung}
Plane in Simulation, bestimme Anfahrspunkt und Anfahrtsrichtung\\
Hand nähert sich, bis Kontakt und Finger schließen sich\\
Evaluation des Kontaktes\\


\subsubsection{Zufallsbasierte Vorwärts-Greifplanung}
Erzeuge Zufällig Griffe\\
Ermittle Kontakt\\
Evaluiere Hypothesen\\

\subsubsection{Griffsynthese auf Objektteilen}
Bei komplexer Geometrie schwierig einen Griff zu berechnen\\
Also berechne Kandidaten auf Teilen eines Objekts\\

\textbf{Mit Formprimitiven}\\
Bestimme Formprimitive aus Objektdimensionen und finde dadurch Griffe\\

\textbf{Box-basiert}\\
Approximiere Objektgeometrie durch Boxen\\
Erzeuge Hypothesen für Boxen\\

\textbf{Mit Superquadriken}\\
Parametrisierte Funktionen, die die Form des geometrischen Objektes definieren\\
Griffhypothese für jede Superquadrik\\

\textbf{Mit medialen Achsen}\\
Untersuche nur \enquote{geometrisch sinnvolle} Griffe\\
Mediale Achse ist Vereinigung der Mittelpunkte aller enthaltenen Kugeln\\
Beschreibt topologisches Skelett des Objekts\\


