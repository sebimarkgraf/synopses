
%%% Local Variables:
%%% mode: latex
%%% TeX-master: "robotik"
%%% End:

\section{Teilsysteme}

\subsection{Mechanische Komponenten}
\subsubsection{Gelenktypen}
\textbf{Rotationsgelenk (R)}\\
Besteht aus Eingang und dazu drehbarem Ausgang.\\
Drehachse in rechten Winkel mit den Achsen der beiden Glieder\\
Ellenbogen\\

\textbf{Torsionsgelenk (T)}\\
Drehachse parallel zu Achsen der beiden Glieder.\\
Unterarm\\

\textbf{Revolvergelenk (V)}\\
Einggangglied parallel zur Drehachse, Ausgangsglied im rechten Winkel dazu.\\
Schultergelenk\\

\textbf{Lineargelenk (L)}\\
Gleitende / Fortschreitende Bewegung\\
Translationsgelenk / Schubgelenk

\subsubsection{Arbeitsraum}
Jene Punkte im 3D-Raum die von der Roboterhand angefahren werden können. Dazu sind 3 Freiheitsgrade in der
Bewegung erforderlich, also mindestens drei Gelenke erforderlich.\\
Grundform: Ignoriere Kollisionen und Winkelbegrenzungen

\subsubsection{Radkonfigurationen}
\textbf{Differentialantrieb}\\
Geradeaus- \& Kurvenfahrten, Drehen auf der Stelle\\
+ einfach\\
- Radregelung in Echtzeit\\

\textbf{Dreirad-Antrieb}\\
Geradeaus- \& Kurvenfahrten, Vorwärts/Rückwärts unterschiedlich.\\
+ einfache Mechanik\\
- eingeschränkt\\

\textbf{Synchro-Antrieb}\\
Geradeaus- \& Kurvenfahrten, Vorwärts/Rückwärts identisch, Plattform dreht nicht mit\\
+ Einfache Regelung\\
+ Geradeaus mechanisch garantiert\\
- Mechanisch Komplex\\

\textbf{Mecanum-Antrieb}\\
+ Uneingeschränkte Beweglichkeit\\
- Mechanisch komplex\\ 
- Aufwendige Regelung\\