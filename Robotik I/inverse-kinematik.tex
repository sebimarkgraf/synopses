
%%% Local Variables:
%%% mode: latex
%%% TeX-master: "robotik"
%%% End:

\section{Inverse Kinematik}%
\label{ik:sec:inverse-kinematik}
Bestimmen der Gelenkwinkelstellungen zu einer gewünschten Lage des Endeffektors\\
Vorgehensweise: Aufstellen der kinematischen Gleichung und lösen nach den Winkeln.

\subsection{Eindeutigkeit}%
\label{ik:sub:eindeutigkeit}
In der Ebene für Systeme mit mehr als 3 Bewegungsfreiheitsgrade mehrere Möglichkeiten. Damit nicht eindeutig!\\
Bei mehr als 6 Bewegungsfreiheitsgraden auch in \SE uneindeutig.

\subsection{Geometrische Methode}%
\label{ik:sub:geometrische-methode}
Nutzen von geometrischen Beziehungen um die Gelenkwinkel aus der Pose zu bestimmen.\\
Kein Verwenden des kinematischen Modells.\\
z.B. verwende Sinus-/Kosinussatz mit Darstellung: 
\begin{align*}
  \cos \theta = \frac{1 - u^2}{1 + u^2}\\
  \sin \theta = \frac{2u}{1 + u^2}
\end{align*}

\subsection{Algebraische Methode}%
\label{ik:sub:algebraische-methode}
Setze TCP Pose \(P_{TCP}\) und Transformation \({}^{BKS}T_{TCP}\)
\[P_{TCP} = {}^{BKS}T_{TCP}(\theta)\]
und führe Koeffizientenvergleich durch.
\[\myMatrix{a_{11} & \cdots & a_{1n}\\ \vdots & \ddots & \vdots \\ a_{n1} & \cdots & a_{nn} } =
  \myMatrix{b_{11} & \cdots & b_{1n}\\ \vdots & \ddots & \vdots \\ b_{n1} & \cdots & b_{nn} } \Rightarrow a_{ij} = b_{ij}\]
Ergibt 12 nicht-triviale Gleichungen.\\
Schwierig zu lösen, deswegen Lösen der Gleichungen Matrix für Matrix anstatt der gesamten Transformation auf einmal.

\subsection{Numerische Methoden}%
\label{ik:sub:numerische-methoden}
Durch Invertieren der Jacobimatrix und Annähern der Vorwärtskinematix
\[\Delta x = f(\Delta \theta) \approx J_f(\theta) \cdot \Delta \theta\]
Dann suche Lösung für Inverse:
\[\Delta \theta \approx F(\Delta x) = J_f^{-1}(\theta) \cdot \Delta x\]

Da Jacobimatrix im Allgemeinen nicht invertierbar: Verwende Pseudoinverse
\[A^+ = A^T{(AA^T)}^{-1}\]

Also Vorgehen
\begin{enumerate}
\item Vorwärtskinematik \(x(t) = f(\theta(t))\)
\item Ableitung nach Zeit \(\frac{dx(t)}{dt} = \dot{x}(t) = J_f(\theta)\dot{\theta}(t)\)
\item Übergang zu Differenzenquotienten \(\Delta x \approx J_f(\theta)\Delta\theta\)
\item Umkehrung \(\Delta\theta \approx J_f^+(\theta)\Delta x\)
\end{enumerate}

\textbf{Iteratives Vorgehen}\\
Pseudoinverse in der Nähe von Singularitäten instabil \(\rightarrow\) Verwende Damped least squares.\\
Minimiere:
\[\min_{\Delta\theta}||J_f(\theta)\Delta\theta - \Delta x||^2_2 + \lambda^2||\Delta\theta||^2_2\]
mit Dämpfung \(\lambda > 0\)