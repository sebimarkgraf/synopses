
%%% Local Variables:
%%% mode: latex
%%% TeX-master: "robotik"
%%% End:

\section{Dynamik}
Untersuchung der Bewegung von Körpern als Folge der auf sie wirkenden Kräfte und Momente.

\subsection{Dynamisches Modell}
Das dynamische Modell beschreibt den Zusammenhang von Kräften, Momenten und Bewegungen, welche in einem mechanischen
Mehrkörpersystem auftreten.\\

Zweck: Analyse der Dynamik, Synthese mechanischer Strukturen, Modellierung elasticher Strukturen, Reglerentwurf

\subsection{Allgemein}
Roboter besteht aus \(n\) Partikeln mit Masse \(m_i\) und Position \(r_i\)\\
Partikel können sich wegen Verbindungen und Gelenken nicht unabhängig voneinander bewegen.\\
\(\rightarrow\) Einführung von Constraints der Form \(g_j(r_1, \ldots, r_n) = 0\)\\

Parameteranzahl: \(3n + k\), aber tatsächlich \(3n-k\) Freiheitsgerade.

\subsection{Generalisierte Koordinaten}
Minimaler Satz an unabhängigen Koordinaten, der den aktuellen Systemzustand vollständig beschreibt.
Finde Koordinaten, welche gleichzeitig die Constraints einhalten.


\subsection{Bewegungsgleichung}
\[\tau = M(q) \cdot \ddot{q} + c(\dot{q}, q) + g(q)\]
\makebox[3cm][l]{\(\tau\): \(n \times 1\)} Vektor der generalisierten Kräfte\\
\makebox[3cm][l]{\(M(q)\): \(n \times n\)} Massenträgheitsmatrix\\
\makebox[3cm][l]{\(c(\dot{q}, q)\): \(x \times 1\)} Vektor der Zentripetal- und Corioliskomponenten\\
\makebox[3cm][l]{\(g(q)\): \(n \times 1 \)} Vektor der Gravitationskomponenten\\
\makebox[3cm][l]{\(q, \dot{q}, \ddot{q}\): \(n \times 1\)} Vektor der generalisierten Koordinaten

\subsection{Probleme}
\textbf{Direktes Problem}:\\
Gegeben Randbedingungen + Kräfte \(\rightarrow\) Generalsierte Koordinaten\\

\textbf{Inverses Problem}:\\
Gegeben Generalisierte Koordiaten \(\rightarrow\) Kräfte

\subsection{Lagrange}
Lagrange-Funktion:
\[ L(q, \dot{q}) = E_{\mathit{kin}}(q, \dot{q}) - E_{\mathit{pot}}(q) \]
Bewegungsgleichungen für jede Koordinate:
\[\tau_i = \frac{d}{dt} \left(\frac{\delta L}{\delta \dot{q}_i}\right) - \frac{\delta L}{\delta q_i}\]

\subsection{Newton-Euler}
Bestimme Kraft nach zweitem Newtonschen Gesetz
\[F_i = \frac{d}{dt} (m_i\ v_{s,i}) = m_i \dot{v}_{s,i}\]
und Drehmoment
\[N_i = \frac{d}{dt}(I_i\ \omega_{s,i}) = I_i \dot{\omega}_{s, i}\]
Beschleunigungen eines Armelements \(i\) hängen vom vorhergenden Armelement ab.\\
\(\rightarrow\) Berechne Beschleunigungen rekursiv von Basis zum Greifer.\\
Kraft und Drehmoment vom nachfolgenden Armelement\\
\(\rightarrow\) Berechne rekursiv vom Greifer zur Basis\\

\(\Rightarrow\) Führt auf Vorwärts- und Rückwärtsgleichungen


\subsection{Herausforderungen}
Modelle nur Approximation, da Störgrößen wie Reibung nicht exakt berechnet werden können.