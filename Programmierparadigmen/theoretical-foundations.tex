
%%% Local Variables:
%%% mode: latex
%%% TeX-master: "propa"
%%% End:

\section{Theoretische Grundlagen}

\subsection{Kalküle}
\begin{itemize}
  \item Minimalistische Programmiersprachen zur Beschreibung von Berechnungen
  \item Zum Führen von Beweisen
\end{itemize}
In dieser Vorlesung \(\lambda\)-Kalkül für sequentielle Sprachen

\subsection{untypisiertes \(\lambda\)-Kalkül}
\begin{tabular}{l l l}
  \textbf{Bezeichnung} & \textbf{Notation} & \textbf{Beispiele}\\
  Variablen & \(x\) & \code{x y}\\
  Abstraktion & \(\lambda x.t\) & \code{\(\lambda\)y. 0}\\
  Funktionsanwendung & \(t_1 t_2\) & \code{f 42}
\end{tabular}\\\\
Die Funktionsanwendung ist \textit{linksassoziativ}: \code{\(\lamda\)x. f x y = \(\lambda\)x ((f x)y: \code{\(\lamda\)x. f x y = \(\lambda\)x ((f x)y)}

   