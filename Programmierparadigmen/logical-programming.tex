
%%% Local Variables:
%%% mode: latex
%%% TeX-master: "propa"
%%% End:

\section{Logische Programmierung}

\subsection{Prolog Syntax}
\textbf{Logische Programmierung}\\
Definiere Objekte und deren Beziehung, wird als Terme einer Termalgebra dargestellt.\\
\textbf{Fakten in Prolog}\\
Einzelne Fakten werden per \enquote{.} definiert.
Programmierer ist verantwortlich für deren Korrektheit.
\textbf{Termsyntax}
\begin{itemize}
  \item Atome: \code{hans, inge,...}
  \item Zahlen: 3, 4.5
  \item Variablen: X, Y, _X, Fisch
  \item Term-Listen: 3,4.5,..
  \item Zusammengesetzt: liebt(fritz, fisch)
\end{itemize}
Atome stehen nur für sich selbst. Variablen sind hingegen Platzhalter für unbekannte Terme.\\
\textbf{Abfragen}\\
Alle Fakten in Datenbank zur Laufzeit. Abfragen daran können mit \enquote{?} eingeleitet werden.
z.B. \code{?liebt(fritz, fisch).}

